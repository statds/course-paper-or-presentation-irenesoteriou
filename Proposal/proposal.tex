\documentclass{article}
\usepackage[utf8]{inputenc}

\title{Research Proposal for STAT 3494W}
\author{Irene Soteriou}
\date{October 2022}

\begin{document}

\maketitle

\section*{Freedom of Speech and Physical Integrity Rights}

\section{Introduction}

For centuries, human beings have philosophized about which rights are integral to a functioning society, and which are necessary to ensure an adequate standard of living for those within it. Critical amongst these rights have always been freedom of speech and physical integrity rights. Over the centuries, a plethora of literature has emerged to explore the relationship between these entitlements, from Piazza and Walsh's 'Physical Integrity Rights and Terrorism' to Haschke's 'Democracy and the Human Right to the Physical Integrity of the Person.' By means of rudimentary statistical analysis, this paper will seek to explore the nature of any correlation, should it be found to exist, between these two fundamental rights. 

\section{Specific Aims}
This paper postulates that freedom of speech and physical integrity rights will be found to be positively correlated. The author of this paper anticipates that this relationship will be strong, with countries that place limits on freedom of speech exhibiting more severe infringements on physical integrity rights. Answering this question may provide valuable insight into the ways in which societies should aim to structure their governments and the rights provided by their legal systems so as to ensure that citizens enjoy a high quality of life. Furthermore, investigation of the aforementioned hypothesis may allow researchers to better assess negative trends in countries experiencing governmental changes, and to more astutely prevent more dangerous developments.

\section{Data Description}
This paper will use as basis for investigation the CIRI Human Rights Dataset, which contains standards-based quantitative data on government adherence to numerous internationally-recognized human rights. This dataset accounts for countries of all regime-types and includes information from all regions of the world. For the purpose of this investigation, this paper will compare data collected from four countries: Afghanistan, Germany, Iran, and the United States of America. For each country, 31 observations are used for each of two variables, with the first variable representing freedom of speech and the second representing physical integrity rights. For the purposes of the subsequent analysis, physical integrity rights will be defined as the rights not to be tortured, summarily executed, disappeared, or imprisoned for political beliefs.

\section{Methods}
To assess the relationship between the two variables -- freedom of speech and physical integrity rights -- the statistical analysis in this paper will employ a simple linear regression for each of four sets of observations. Accordingly, one simple linear regression will be used to assess the strength of the relationship between freedom of speech and physical integrity rights in Afghanistan, a second simple linear regression will be used to assess the strength of the relationship between these two variables as observed in Germany, a third for Iran, and a fourth for the United States of America. This method will help reveal to what degree, if any, a relationship exists.


\section{Discussion}
I expect to find a strong positive correlation between freedom of speech and physical integrity rights. It appears reasonable to assert that restrictions on freedom of speech may often occur in countries that place greater restrictions on individual freedoms overall, and such countries tend to experience higher rates of infringement on physical rights as well. Likewise, countries that have greater freedom of speech may typically have greater infrastructure in place for public opposition to the potential for infringement on physical integrity rights. A strong positive correlation between the two variables will reinforce existing assumptions about the influence and importance of freedom of speech as a means of preventing governmental tyranny and infringement of rights. If the results of the investigation appear to contradict these assumptions, then we will have reason to reevaluate these baseline principles. This could have significant impact on views within political science and legal theory.

\section{Conclusion}
In conclusion, this paper will seek to investigate the relationship between freedom of speech and physical integrity rights in four countries, some of which have a reputation for poor overall quality of life and others of which have a reputation for relatively high quality of life. By investigating the nature of this relationship, should it exist, this paper aims to provide insight into the direction of legal theory and political thought.
\end{document}
